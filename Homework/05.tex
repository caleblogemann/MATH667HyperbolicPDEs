\documentclass[11pt, oneside]{article}
\usepackage[letterpaper, margin=2cm]{geometry}
\usepackage{MATH667}

\begin{document}
\noindent \textbf{\Large{Caleb Logemann \\
MATH667 Hyperbolic Partial Differential Equations \\
Homework 5
}}

%\lstinputlisting[language=MATLAB]{H01_23.m}
\begin{enumerate}
  \item % #1
    For the following schemes to solve nonlinear conservation laws, show which
    ones are monotone schemes.
    \begin{itemize}
      \item Godunov scheme


      \item Lax-Friedrichs scheme \hfill \\ % Done 
        The Lax-Friedrichs scheme is monotone if the CFL condition is meet.
        To see this note that the Lax-Friedrichs method can be written as
        $U^{n+1}_j = H(U^n_{j-1}, U^n_j, U^n_{j-1})$ where
        \[
          H(U_{j-1}, U_j, U_{j+1}) = \p{1 - \frac{\alpha \Delta t}{\Delta x}} U_j 
          + \frac{\alpha \Delta t}{2 \Delta x} \p{U_{j+1} + U_{j-1}} 
          - \frac{\Delta t}{2 \Delta x} \p{f(U_{j+1}) - f(U_{j-1})}
        \]
        and $\alpha = \max[u]{\abs{f'(u)}}$
        To show that this method is monotone, we need to show that
        $\pd{H}{U_i} \ge 0$ for $i = j-1, j, j+1$.

        First consider $\pd{H}{U_j}$.
        \begin{align*}
          \pd{H}{U_j} &= \p{1 - \frac{\alpha \Delta t}{\Delta x}} \\
          \p{1 - \frac{\alpha \Delta t}{\Delta x}} &\ge 0 \\
          1 &\ge \frac{\alpha \Delta t}{\Delta x} \\
        \end{align*}
        This is exactly the CFL condition, so this condition is satisfied by this method.

        Second consider $\pd{H}{U_{j-1}}$.
        \begin{align*}
          \pd{H}{U_{j-1}} &= \frac{\alpha \Delta t}{2 \Delta x} + \frac{\Delta t}{2 \Delta x} f'(U_{j-1}) \\
          &= \frac{\Delta t}{2 \Delta x}\p{\alpha + f'(U_{j-1})}
          \intertext{Since $\alpha = \max[u]{\abs{f'(u)}} \ge f'(U_{j-1})$, then
            $\p{\alpha + f'(U_{j-1})} \ge 0$ and}
          \pd{H}{U_{j-1}} &\ge 0
        \end{align*}

        Finally consider $\pd{H}{U_{j+1}}$.
        \begin{align*}
          \pd{H}{U_{j+1}} &= \frac{\alpha \Delta t}{2 \Delta x} - \frac{\Delta t}{2 \Delta x} f'(U_{j+1}) \\
          &= \frac{\Delta t}{2 \Delta x}\p{\alpha - f'(U_{j+1})}
          \intertext{Since $\alpha = \max[u]{\abs{f'(u)}} \ge f'(U_{j-1})$, then
            $\p{\alpha - f'(U_{j+1})} \ge 0$ and}
          \pd{H}{U_{j+1}} &\ge 0
        \end{align*}

        These three conditions are met by the Lax-Friedrichs method, so the scheme is monotone.

      \item Local Lax-Friedrichs scheme \hfill \\ % Done
        The Local Lax-Friedrichs scheme is monotone if the CFL condition is meet.
        To see this note that the Local Lax-Friedrichs method can be written as
        $U^{n+1}_j = H(U^n_{j-1}, U^n_j, U^n_{j-1})$ where
        \[
          H(U_{j-1}, U_j, U_{j+1}) = \p{1 - \frac{\p{\alpha_+ + \alpha_-} \Delta t}{2\Delta x}} U_j
          + \frac{\Delta t}{2 \Delta x} \p{\alpha_+ U_{j+1} + \alpha_- U_{j-1}} 
          - \frac{\Delta t}{2 \Delta x} \p{f(U_{j+1}) - f(U_{j-1})}
        \]
        and $\alpha_+ = \max[\p{U_j, U_{j+1}}]{\abs{f'(u)}}$ and $\alpha_- = \max[\p{U_j, U_{j-1}}]{\abs{f'(u)}}$.
        To show that this method is monotone, we need to show that
        $\pd{H}{U_i} \ge 0$ for $i = j-1, j, j+1$.

        First consider $\pd{H}{U_j}$.
        \begin{align*}
          \pd{H}{U_j} &= \p{1 - \frac{\p{\alpha_+ + \alpha_-} \Delta t}{2\Delta x}} \\
          \p{1 - \frac{\p{\alpha_+ + \alpha_-} \Delta t}{2\Delta x}} &\ge 0 \\
          1 &\ge \frac{\p{\alpha_+ + \alpha_-} \Delta t}{2\Delta x} \\
        \end{align*}
        Since $\alpha_+$ and $\alpha_-$ are both less than $\alpha$, this
        condition is met if the CFL condition is met.

        Second consider $\pd{H}{U_{j-1}}$.
        \begin{align*}
          \pd{H}{U_{j-1}} &= \frac{\alpha_- \Delta t}{2 \Delta x} + \frac{\Delta t}{2 \Delta x} f'(U_{j-1}) \\
          &= \frac{\Delta t}{2 \Delta x}\p{\alpha_- + f'(U_{j-1})}
          \intertext{Since $\alpha_- = \max[\br{U_{j-1}, U_j}]{\abs{f'(u)}} \ge f'(U_{j-1})$, then
            $\p{\alpha_- + f'(U_{j-1})} \ge 0$ and}
          \pd{H}{U_{j-1}} &\ge 0
        \end{align*}

        Finally consider $\pd{H}{U_{j+1}}$.
        \begin{align*}
          \pd{H}{U_{j+1}} &= \frac{\alpha_+ \Delta t}{2 \Delta x} - \frac{\Delta t}{2 \Delta x} f'(U_{j+1}) \\
          &= \frac{\Delta t}{2 \Delta x}\p{\alpha_+ - f'(U_{j+1})}
          \intertext{Since $\alpha = \max[\br{U_j, U_{j+1}}]{\abs{f'(u)}} \ge f'(U_{j-1})$, then
            $\p{\alpha_+ - f'(U_{j+1})} \ge 0$ and}
          \pd{H}{U_{j+1}} &\ge 0
        \end{align*}

        These three conditions are met by the Local Lax-Friedrichs method, so the scheme is monotone.

      \item Lax-Wendroff scheme \hfill \\ % Done
        The Lax-Wendroff scheme is not monotone.
        This is obvious because the Lax-Wendroff scheme is second order and
        monotone schemes must be first order at most.
        Also Lax-Wendroff creates oscillations at shocks, which are clearly not
        monotone.

        To see this specifically, consider $f(u) = u$.
        In this case,
        \[
          H(U_{j-1}, U_j, U_{j+1}) = U_j - \frac{\Delta t}{2\Delta x} \p{U_{j+1} - U_{j-1}}
          + \frac{\Delta t^2}{2 \Delta x^2} \p{U_{j+1} - 2U_j + U_{j-1}}
        \]
        Now consider $\pd{H}{U_{j+1}}$
        \begin{align*}
          \pd{H}{U_{j+1}} &= -\frac{\Delta t}{2\Delta x} + \frac{\Delta t^2}{2 \Delta x^2} \\
          &= \frac{\Delta t}{2\Delta x}\p{\frac{\Delta t}{\Delta x} - 1} \\
          &\le 0 \\
        \end{align*}
        If the CFL condition is satisfied then
        $\p{\frac{\Delta t}{\Delta x} - 1} \le 0$, so this partial
        derivative is negative and the method is not monotone.
    \end{itemize}

  \item % #2
    Solve Burger's equation $u_t + \p{\frac{u^2}{2}}_x = 0$ on
    $x \in \br{0, 2\pi}$ with initial data $u(x, 0) = 1 + \frac{1}{2} \sin{x}$.
    Let's consider the 1\textsuperscript{st} order finite difference Godunov scheme.
    Implement the scheme to (a) $t = 1.0$ and (b) $t = 3.0$.
    Apply periodic boundary conditions.
    For part (a) output $L^{\infty}$ error/order table with uniform mesh
    $N = 20, 40, 80, 160$.
    For part (b) graph the simulation with $N = 80$ and solid line for exact
    solution and symbols for numerical approximations.
\end{enumerate}
\end{document}
